\documentclass[12pt,letterpaper]{article}
\usepackage{amsfonts, amssymb, amsmath}
\usepackage[usenames,dvipsnames]{color}
\newcommand{\stnote}[1]{\textcolor{Blue}{\textbf{ST: #1}}}
\newcommand{\menote}[1]{\textcolor{Red}{\textbf{ME: #1}}}

\begin{document}
\begin{center}\textbf{\underline{Bayes Filter for Gesture Recognition}}\end{center}
\textbf{Definitions}
\begin{itemize}
\item{Hidden State Space}
	\begin{itemize}
	\item{$\mathcal{X}$: the set of $m$ objects in the scene that can be referenced and a state for no item being referenced}
	\item{An example would be: $x = \text{Object 1 is being referenced}$}
	\item{Each object in the scene has an associated set of three dimensional points ($x_p$) and set of descriptor keywords $(x_d)$}
	\end{itemize}
\item{Observations}
	\begin{itemize}
	\item{$\mathcal{Z}$: the set of four-tuple $\{l, r, h, s\}$}
	\item{$l$ represents the observed origin ($l_o$) and vector ($l_v$) for the left arm}
	\item{$r$ represents the observed origin  ($r_o$) and vector ($r_v$)  for the right arm}
	\item{$h$ represents the observed origin  ($h_o$) and vector ($h_v$)  for head}
	\item{$s$ represents the observed speech from the user, consisting of a list of words}
	\end{itemize}
\item{Transition Function}
	\begin{itemize}
	\item{$\mathcal{T}$: a function such that $\mathcal{T}(x_a, x_b)$ is equivalent to the probability that $x_a$ transitions to $x_b$}
	\item{\stnote{Model with Poisson?}}
	\end{itemize}
\item{$\mathcal{N}$: a function that returns the probability of a sample under a Gaussian distribution given a mean and variance}
	\begin{itemize}
	\item{Applied as $\mathcal{N}(\mu, \sigma, \text{sample})$}
	\end{itemize}
\item{$\Phi$: a function that, given an origin and two points, returns the angle between the two points}
	\begin{itemize}
	\item{Applied as $\Phi(\text{origin}, p_1, p_2)$}
	\end{itemize}
\item{$\mathcal{I}$: an indicator function applied as $\mathcal{I}(\text{word}, \text{corpus}$ that returns 1 if the word is in the corpus and 0 otherwise}
%\item{$\mu_l$, $\mu_r$, \mu_h$: the sample means for each component of the observations}
%\item{$\sigma_l$, $\sigma_r$, $\sigma_h$: the sample variances for each component of the observations}
\end{itemize}
\textbf{Equations}
\begin{itemize}
\item{Time Update}
	\begin{itemize}
	\item{An equation used to determine the probability that $\mathcal{X}_t = x$ given only previous belief states}
	\item{$P(\mathcal{X}_t = x | \mathcal{X}_{t-1} .... \mathcal{X}_0) =\displaystyle\sum_{x' \in \mathcal{X}} \mathcal{T}(x, x')*bel(\mathcal{X}_{t-1} = x')$}
	\end{itemize}
\item{Measurement Update}
	\begin{itemize}
	\item{An equation used to determine the belief that $\mathcal{X}_t = x$ given observation $\mathcal{Z}_{t-1}$}
	\item{$P(\mathcal{X}_t=x | \mathcal{Z}_{t-1}) = P(X_t=x | l_{t-1})*P(X_t=x | r_{t-1})*P(X_t=x |h_{t-1})*P(X_t=x | s_{t-1})$}
	\item{$P(\mathcal{X}_t=x|l_{t-1}) = \displaystyle \prod_{p \in x_p} \mathcal{N}(\mu_l=0, \sigma_l, \Phi(l_o,l_v, p))$\\}
	\item{$P(\mathcal{X}_t=x|r_{t-1}) = \displaystyle \prod_{p \in x_p} \mathcal{N}(\mu_r=0, \sigma_r, \Phi(r_o,r_v, p))$\\}
	\item{$P(\mathcal{X}_t=x|h_{t-1}) = \displaystyle \prod_{p \in x_p} \mathcal{N}(\mu_h=0, \sigma_h, \Phi(h_o,h_v, p))$\\}
	\item{$P(\mathcal{X}_t=x|s_{t-1}) = \frac{\displaystyle\sum_{w\in s_{t-1}} \mathcal{I}(w, x_d)}{\displaystyle\sum_{x \in \mathcal{X}}\sum_{w\in s_{t-1}} \mathcal{I}(w, x'_d)}$}
	\end{itemize}
\item{Belief Update}
	\begin{itemize}
	\item{An equation that produces the probability that $\mathcal{X}_t = x$ given all past observations and belief states, namely the product of the measurement and time updates}
	\item{$bel(X_t = x) = P(\mathcal{X}_t=x | \mathcal{Z}_{t-1})*P(\mathcal{X}_t = x | \mathcal{X}_{t-1} .... \mathcal{X}_0)$}
	\item{$bel(\mathcal{X}_0 = x) =  \frac{1}{m+1}$ (uniform initialization of belief)}
	\end{itemize}
\end{itemize}
\end{document}