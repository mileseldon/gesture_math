\documentclass[letterpaper, 10 pt, conference]{ieeeconf} 
\IEEEoverridecommandlockouts 
\overrideIEEEmargins                                      % Needed to meet printer requirements.
\usepackage{natbib}


\title{\LARGE \bf
Interpreting Multimodal Referring Expressions in Real Time}
\author{Miles Eldon$^{1}$ and Stefanie Tellex$^{1}$
\thanks{$^{1}$Computer Science Department, Brown University}
}

\usepackage{amsfonts, amssymb, amsmath}
\usepackage[usenames,dvipsnames]{color}
\newcommand{\stnote}[1]{\textcolor{Blue}{\textbf{ST: #1}}}
\newcommand{\menote}[1]{\textcolor{Red}{\textbf{ME: #1}}}

\begin{document}

\maketitle
\thispagestyle{empty}
\pagestyle{empty}

\begin{abstract}
Identifying objects for shared tasks, such as a knife for assistance
at cooking, or a screw used to assemble a part on a factory floor, is
a key part of many human-robot collaborative tasks.  Robots that
collaborate with people must be able to understand their references to
objects in the environment.  Existing work has addressed this problem
in single modalities, such as natural language or gesture, but a gap
remains in creating real-time multimodal systems that simultaneously
fuse information from language and gesture in a principled mathemtical
framework.  We define a multimodal Bayes' filtering approach to
interpreting referring expressions to object using language and
gesture.  We collected a new RGB-D and audio dataset of people
referring to objects in a tabletop setting and demonstrate that our
approach successfully integrates information from language and gesture
in real time to quickly and accurately identify objects.
\end{abstract}

\section{INTRODUCTION}

\section{RELATED WORK}


\citep{matuszek12}

\section{TECHNICAL APPROACH}
Our aim is to estimate a distribution over the object that a person is
referring to given language and gesture inputs.  We frame the problem
as a Bayes' filter, where the hidden state, $\mathcal{X}$, is the set
of $m$ objects in the scene that can be referenced. The robot observes the person's actions and
speech, $\mathcal{Z}$, and at each timestep estimates a distribution
over $\mathcal{X}$:
\begin{align}
  p(x_t | z_0 \dots z_{0:t})
\end{align}




The time update updates the belief that the user is in any specific state given all previous information:
\begin{align}
p(x_t | z_{0:t-1}) = \int p(x_t|x_{t-1})\times p(x_{t-1} | \mathcal{Z}_{0:t-1}) \text{d}x_{t-1}
\end{align}

The measurement update combines the previous belief with the newest observation to update each belief state: 
\begin{align}
p(x_t | \mathcal{Z}_{0:t}) = \frac{p(z_t | x_t) \times p(x_t | z_{0:t-1})}{p(z_t | z_{0:t-1})} \\\propto p(z_t | x_t) \times p(x_t | z_{0:t-1})
\end{align}





\subsection{Observation Model}

We assume access to an observation model of the form:
\begin{align}
p(z_t | x_t)
\end{align}

Observations consist of a tuple consisting of a person's actions,
$\langle l, r, h, s\rangle $ where:
\begin{itemize}
	\item $l$ represents the observed origin ($l_o$) and vector ($l_v$) for the left arm.
	\item $r$ represents the observed origin  ($r_o$) and vector ($r_v$)  for the right arm .
	\item $h$ represents the observed origin  ($h_o$) and vector ($h_v$)  for head.
	\item $s$ represents the observed speech from the user, consisting of a list of words.
	\end{itemize}

\begin{align}
p(z_t | x_t) = p(l, r, h, s | x_t)\\
p(z_t | x_t) = p(l | x_t) \times p(r | x_t) \times p(h | x_t) \times p(s | x_t)
\end{align}
{\bf Gesture.}  We model gesture as a vector through three dimensional space. We calculate the angle between the gesture vector and the vector from the gesture origin to the mean of each cluster, and then use the PDF of a gaussian with trained variance to determine the weight that should be assigned to that object. Let $\Phi(<origin>, <point>, <point>)$ give the angle between the two points with the given origin.
%We calculate the probability of a gesture by examining every three dimensional particle (denoted as $q$) in an object and calculating the angle (function $\Phi$) between the vector formed by the gesture and the vector formed with that particle. We then use a Gaussian distribution with a variance ($\sigma$) found during training to calculate the probability of seeing that angle difference. We then take the product of each of these points and normalize it so that small objects with few particles don't always dominate the probability distribution. The probability of each gesture given the state is as follows:
\begin{align}
p(l | x_t) \propto \Phi(l_o, l_v, x_t)\\
p(r | x_t) \propto \Phi(r_o, r_v, x_t)
%p(l|x_t) =[\displaystyle \prod_{q \in x_t} \mathcal{N}(\mu_l=0, \sigma_l, \Phi(l_o,l_v, q))]^{(\frac{\sum_{x'\in\mathcal{X}} len(x'_q)}{len(x_q)})}\\
%p(r|x_t) = [\displaystyle \prod_{q \in x_t} \mathcal{N}(\mu_r=0, \sigma_r, \Phi(r_o,r_v, q))]^{(\frac{\sum_{x'\in\mathcal{X}} len(x'_q)}{len(x_q)})}
\end{align}
{\bf Head Pose.}
Head pose is modeled in the same manner as arm gestures.
\begin{align}
p(h | x_t) \propto \Phi(h_o, h_v, x_t)
%p(h|x_t) = [\displaystyle \prod_{q \in x_t} \mathcal{N}(\mu_h=0, \sigma_h, \Phi(h_o,h_v, q))]^{(\frac{\sum_{x'\in\mathcal{X}} len(x'_q)}{len(x_q)})}
\end{align}
\menote{More concise way to show this? They are all the same besides variance. I guess we could just do $\prod_{g\in\{h,l,r\}}$}
{\bf Speech.}
We model speech with a simple bag of words model. We take the words in a given speech input and count how many words in this text match descriptors (denoted  $x_d$) of specific objects.
\begin{align}
p(s |x_t) = \displaystyle \prod_{w \in s} p(w | x_t)
\end{align}


\section{EVALUATION}

\section{CONCLUSION}

\section{REFERENCES}
\bibliographystyle{abbrvnat}
\bibliography{main}



\end{document}
