\documentclass[12pt,letterpaper]{article}
\usepackage{amsfonts, amssymb, amsmath}
\usepackage[usenames,dvipsnames]{color}
\newcommand{\stnote}[1]{\textcolor{Blue}{\textbf{ST: #1}}}

\begin{document}
\begin{center}\textbf{\underline{Bayes Filter for Gesture Recognition}}\end{center}
\textbf{Variables}\\\\
\textit{Hidden State Space}\\
$\mathbb{Z} = \{z^t_\varnothing, z^t_1, ...., z^t_m\}$\\
Where there are $m$ objects and $t$ represents the timestamp on the state.\\
A state consists of an object being referenced (or no object being referenced) and a timestamp.\\
Each state $z^t_i$ consists of a series of 3D points.\\
NOTE: Consider combinations of objects? This would allow a good model of "The yellow objects". Initial model is for single objects only, but that is definitely a worthwhile extension.\\\\
\textit{Observations}\\
$\mathbb{O} = \{o_1, .... o_t\}$\\
Where the subscript represents the timestamp for each observation.\\
Where $o_i= \{l_i,r_i, h_i, s_i\}$\\
Each observation consists of four parts:\\
$l_i$, the left arm vector at time $i$\\
$r_i$, the right arm vector\\
$h_i$, the head vector\\
$s_i$, speech\\
Since not all components of the observation tuple are guaranteed to be present at the same time, any of the four can take on a null value\\\\
\textit{Transition Function}\\
The transition function $\mathbb{T}(z^t_i, z^{t+1}_k)$ returns the probability of state $z^t_i$ transitioning to $z^{t+1}_k$.\\
Transition probabilities should be high for $\mathbb{T}(z_i^t,z^{t+1}_k)$ when $i=k$ and low otherwise. (Or we can start with uniform, but this makes more sense to me intuitively)\\
\stnote{I think we should use maybe a Poisson Process. }
We could also weight the transition functions based on shared properties.\\\\
Let $\mathbb{N}$ represent the probability of seeing a specific sample with a  normal distribution of the specified parameters,  $\theta$ represent the angle between the input vector and the sample point, and w,x, and y be exponents for weighting each sample appropriately.\\\\
\textbf{Equations}\\\\
We wish to know the most likely state (object being referenced) given our observations and previous state estimation, namely:\\
$\underset{z^t_i}{\text{argmax}}[P(z^t_i | o_{t-1})*\displaystyle\sum_{z^{t-1}_k \in \mathbb{Z}} P(z^t_i|z^{t-1}_k)] $\\
Where:\\
$P(z^t_i | z^{t-1}_k) = \mathbb{T}(z^{t-1}_k, z^t_i)*P(z^{t-1}_k)$\\
$P(z^0_k) = \frac{1}{m+1}$\\
$P(z^t_i|o_{t-1}) = P(z^t_i|l_{t-1})^u*P(z^t_i|r_{t-1})^w*P(z^t_i|h_{t-1})^x*P(z^t_i|s_{t-1})^y$\\
$P(z^t_i|l_{t-1}) = \displaystyle \prod_{p \in z^t_i} \mathbb{N}(\mu_l=0, \sigma_l, \theta(l_{t-1}, p))$\\
$P(z^t_i|r_{t-1})=\displaystyle \prod_{p \in z^t_i} \mathbb{N}(\mu_r=0, \sigma_r, \theta(r_{t-1}, p))$\\
$P(z^t_i|h_{t-1}) =\displaystyle \prod_{p \in z^t_i} \mathbb{N}(\mu_h=0, \sigma_h, \theta(h_{t-1}, p))$\\
$P(z^t_i|s_{t-1}) = \text{TBD}$\\
NOTE: How to deal with varying cluster sizes?\\
1) Use only mean instead of product.\\
2) Pad with mean so that all clusters have the same number of particles\\
3) Pad with random sample so that all clusters have the same number of particles

\end{document}
